\fvisa{Internationalen}{Upp trälar uti alla stater}\par
\vspace{10pt}
Upp, trälar uti alla stater,\\
som hungern bojor lagt uppå\\
Det dånar uti rättens krater,\\
snart skall uppbrottets timma slå.\\
Störtas skall det gamla snart i gruset.\\
Slav, stig upp för att slå dig fri!\\
Från mörkret stiga vi mot ljuset,\\
från intet allt vi vilja bli.\par
\vspace{10pt}
Upp till kamp emot kvalen.\\
Siste striden det är,\\
ty Internationalen\\
till alla lycka bär.\\
Upp till kamp emot kvalen.\\
Sista striden det är,\\
ty Internationalen\\
åt alla lycka bär.\par
\vspace{10pt}
I höjden räddarn vi ej hälsa,\\
ej gudar, furstar stå oss bi,\\
nej, själva vilja vi oss frälsa,\\
och samfälld skall vår räddning bli\\
För att kräva ut det stulna, bröder,\\
och för att slita andens band,\\
vi smida medan järnet glöder,\\
med senig arm och kraftig hand.\par
\vspace{10pt}
Upp till kamp emot kvalen...\par
\newpage
I sin förgudning avskyvärda,\\
månn´guldets kungar nånsin haft\\
ett annat mål än att bli närda\\
av proletärens arbetskraft?\\
Vad han skapat under nöd och vaka\\
utav tjuvar rånat är;\\
när folket kräva det tillbaka\\
sin egen rätt de blott begär.\par
\vspace{10pt}
I sin förgudning avskyvärda,\\
Upp till kamp emot kvalen...\par
\vspace{10pt}
I sin förgudning avskyvärda,\\
Båd´stat och lagar oss förtrycka\\
vi under skatter dignar ner.\\
Den rike inga plikter tycka,\\
den arme ingen rätt man ger.\\
Länge nog som myndingar vi böjt oss,\\
jämlikheten skall nu bli lag.\\
Med plikterna vi hittills nöjt oss .\\
Nu taga vi vår rätt en dag.\par
\vspace{10pt}
Upp till kamp emot kvalen...\par
\vfill
\hfill {\footnotesize\textit{forts. $\rightarrow$}}
\newpage
Till krigets slaktande vi dragits,\\
vi mejats ned i jämna led.\\
För furstars lögner har vi slagits,\\
nu vill vi skapa evig fred.\\
Om de oss driver, dessa kanibaler,\\
mot våra grannar än en gång,\\
vi skjuter våra generaler\\
och sjunger broderskapets sång.\par
\vspace{10pt}
Upp till kamp emot kvalen...\par
\vspace{10pt}
Arbetare, i stad på landet,\\
en gång skall jorden bliva vår\\
När fast vi knyta brodersbandet,\\
då lättingen ej råda får.\\
Många rovdjur på vårt blod sig mätta\\
men när vi nu till vårt försvar,\\
en dag en gräns för dessa sätta,\\
skall solen stråla lika klar.\par
\vspace{10pt}
Upp till kamp emot kvalen...
