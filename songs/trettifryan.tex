\fvisa{Trettifyran}{}
\vspace{10pt}
Denna kåk har varit våran\\
uti många herrans år.\\
Denna kåk har varit vår\\
och det har nog satt sina spår.\\
Denna kåk har hängt med\\
och den har stått i vått och torrt,\\
men nu är det slut med det\\
för nu skall trettifyran bort.\par
\vspace{10pt}
Ja nu är det slut på gamla tider,\\
Ja nu är det färdigt inom kort,\\
nu skall hela rasket rivas,\\
nu skall hela rasket bort.\\
Så jag tar farväl\\
och stora tårar rullar på min kind.\\
Nu är det slut på gamla tider,\\
nu går trettifyran i himlen in.\par
\vspace{10pt}
Denna kåk var ganska rar\\
och släppte solen till oss in.\\
Den var också generös med fukt\\
och kyla  regn och vind.\\
Den var snäll och lite gnällig\\
och den ville alla väl.\\
Den var vår i alla väder\\
fastän gisten, ful och skev.\par
\vspace{10pt}
Ja, nu är det slut...\par
\vspace{10pt}
Här i kåken har vi härjat\\
sen vi alla varit små.\\
här i kåken klådde morsan\\
vice värden gul och blå.\\
Ja vår kåk har fått stå pall\\
för smällar hårda så det dög,\\
som när far gick genom väggen\\
så att spån och plankor flög.\par
\vspace{10pt}
Ja, nu är det slut...\par
\vspace{10pt}
{\footnotesize\textit{Text: Olle Adolphson\\
Musik: Stuart Hamblen\par
\vspace{10pt}
Den som tycker att melodin eller texten är tråkig kan byta ut den mot ``Oxdragarsång'', ``Flickorna i Småland'' eller ``Fjäriln vingad''. Se även ``Kungens man''.}}
