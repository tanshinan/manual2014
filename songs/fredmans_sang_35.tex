\fvisa{Fredmans Sång N:o 35}{Gubben Noach, Gubben Noach}
\vspace{10pt}
\revrpt Gubben Noach, Gubben Noach\\ 
Var en hedersman,\rpt \\ 
När han gick ur arken\\ 
Plantera han på marken\\ 
Mycket vin, ja mycket vin, ja\\ 
D etta gjorde han.\\ 
\\ 
\revrpt Noach rodde, Noach rodde\\ 
Ur sin gamla ark,\rpt \\ 
Köpte sig buteljer,\\ 
Sådana man sälljer,\\ 
För at dricka, för at dricka\\ 
På vår nya park.\\ 
\\  
\revrpt Han väl visste, han väl visste\\ 
At en mänska var\rpt \\ 
Torstig af naturen\\ 
Som de andra djuren,\\ 
Därför han ock, därför han ock\\ 
Vin planterat har.\\ 
 \\ 
\revrpt Gumman Noach, Gumman Noach\\ 
Var en heders fru;\rpt \\ 
Hon gaf man sin dricka;\\ 
Fick jag sådan flicka,\\ 
Gifte jag mig, gifte jag mig\\ 
Just på stunden nu.\\ 
 \\ 
\revrpt Aldrig sad' hon, aldrig sad' hon\\ 
Kära far nå nå;\rpt \\ 
Sätt ifrån dig kruset;\\ 
Nej det ena ruset\\ 
På det andra, på det andra\\ 
Lät hon gubben få.\\ 
 \\ 
\revrpt Gubben Noach, Gubben Noach\\ 
Brukte egna hår,\rpt \\ 
Pipskägg, hakan trinder\\ 
Rosenröda kinder,\\ 
Drack i botten, drack i botten.\\ 
Hurra och gutår!\\ 
 \\ 
\revrpt Då var lustigt, då var lustigt\\ 
På vår gröna jord;\rpt \\ 
Man fick väl til bästa,\\ 
Ingen torstig nästa\\ 
Satt och blängde, satt och blängde\\ 
Vid et dukadt bord.\\ 
 \\ 
\revrpt Inga skålar, inga skålar\\ 
Gjorde då besvär,\rpt \\ 
Då var ej den läran:\\ 
Jag skal ha den äran;\\ 
Nej i botten, nej i botten\\ 
Drack man ur så här.\\ 
