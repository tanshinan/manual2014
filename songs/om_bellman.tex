\tvisa{OM BELLMAN}
\vspace{10pt}
\hspace{10pt}I Maria församling föddes Carl Michael Bellman (1740-
1795). Under barndomen och ungdomens strövtåg blev Carl 
Michael förtrogen med det brokiga livet på Ulla Winblads och 
korpral Mollbergs söder, men hans egen barndomsmiljö var inte 
deras. Bellman gick bara en kort tid i Maria skola och 
undervisades sedan av informatorer i hemmet. Familjen var inte 
utan lärdomstradition; Carl Michaels farfar hade varit professor i 
Uppsala och hans far hade valt ämbetsmannavägen först efter 
flera utländska studieår.\par
\hspace{10pt} Att Bellman själv inte kom att leva ett vanligt borgerligt 
liv berodde på hans nöjeslystnad och ovanliga sällskapstalanger. 
1760-talets Stockholm hemsöktes av en svårartad ekonomisk kris 
med hastigt sjunkande penningvärde. De unga sekreterarna i 
ämbetsverkan roade sig på rent trots mot inflationen - vad 
tjänade det till att spara på slantarna som för var dag blev 
allt mindre värda? Hemifrån hade Bellman inte heller någon 
hjälp att vänta. Familjens ekonomi undergrävdes ungefär samtidigt som 
sonen 1763 flydde till Norge från efterhängsna fordringsägare. 
Från och med detta år började hans visor spridas i huvudstaden. 
Snart var han åter bosatt på Söder, och i goda vänners hem 
träffade han stadens unga skalder som han underhöll med sin 
egna märkliga talang. Och det var inte till krogar och kåkar hans 
talang öppnade dörrarna, utan till grevars, riksråd och 
diplomaters palats - och från och med 1772 även till kungens. 
1776 fick han titeln hovsekreterare och uppbar en liten 
``pension'' ur kungens egen handkassa. Samma år tillträdde han en 
någorlunda väl avlönad tjänst vid Nummerlotteriet.\par
\vspace{10pt}
{\footnotesize\textit{Ur ``Litteraturorientering för gymnasieskolan''.}}
