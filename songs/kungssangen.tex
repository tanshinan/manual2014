\fvisa{KUNGSSÅNGEN}{Ur svenska hjärtans djup en gång}
\vspace{10pt}
Ur svenska hjärtans djup en gång\\
en samfälld och en enkel sång,\\
som går till kungen fram!\\
Var honom trofast och hans ätt,\\
gör kronan på hans hjässa lätt,\\
och all din tro till honom sätt,\\
du folk av frejdad stam!\par
\vspace{10pt}
O konung, folkets majestät\\
är även ditt: beskärma det\\
och värna det från fall!\\
Stå oss all världens härar mot,\\
vi blinka ej för deras hot:\\
vi lägga dem inför din fot -\\
en kunglig fotapall.\par
\vspace{10pt}
Men stundar ock vårt fall en dag,\\
från dina skuldror purpurn tag,\\
lyft av dig kronans tvång\\
och drag de kära färger på,\\
det gamla gula och det blå,\\
och med ett svärd i handen gå\\
till kamp och undergång!\par
\newpage
Och grip vår sista fana du\\
och dristeliga för ännu\\
i döden dina män!\\
Ditt trogna folk med hjältemod\\
skall sömma av sitt bästa blod\\
en kunglig purpur varm och god,\\
och svepa dig i den.\par
\vspace{10pt}
Du himlens Herre, med oss var,\\
som förr du med oss varit har,\\
och liva på vår strand\\
det gamla lynnets art igen\\
hos sveakungen och hans män.\\
Och låt din ande vila än\\
utöver nordanland!\par
\vspace{10pt}
{\footnotesize\textit{Text: C. V. A. Strandberg\\
Musik: Otto Lindblad}}\par
\vspace{10pt}
{\footnotesize\textit{Första och sista versen är de som brukar sjungas.}}
