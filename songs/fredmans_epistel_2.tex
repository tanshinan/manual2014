\fvisa{Fredmans Epistel N:o 2}{Nå, skruva fi olen, hej spelman, skynda dig!}
{\footnotesize\textit{Till Fader Berg, rörande fiolen.}}\par
\vspace{10pt}
Nå, skruva fiolen, hej spelman, skynda dig!\\
Kära syster, hej! Svara inte nej,\\
svara ja, så blir vi glada.\\
Sätt dig du på stolen och stryk din silversträng!\\
Röda stråken släng och med armen sväng;\\
gör ej fiolen skada.\\
Du svettas, stor sak, i brännvin ska du bada,\\
ty under detta tak är Bacchi lada\\
- Ganska riktigt, ditt kall är viktigt\\
båd' för öra, syn och smak.\par
\vspace{10pt}
Bland nymfernas skara, är du omistlig man;\\
Du båd’ vill och kan mer än någon ann\\
de unga hjärtan binda,\\
och kärlekens skara på dina strängar står;\\
varje ton du slår, du ett hjärta får\\
att konstigt sammanlinda.\\
Just på en minut små ögon bliva blinda,\\
och fl ickorna till slut de blir så trinda.\\
- Hur du bullrar!\\
Men nymfen kullrar\\
och du skrattar med din trut.\par
\newpage
Jag älskar de sköna, men vinet ändå mer;\\
Jag på båda ser och åt båda ler\\
Men skiljer ändå båda.\\
En Nymf i det gröna och vin i gröna glas:\\
Lika gott kalas, båda om mig dras.\\
Ge stråken mera kåda;\\
Konfonium tag där uti min gröna låda;\\
Och vinet står ju här.\\
Jag är i våda.\\
- Supa, dricka,\\
Och ha sin flicka, är vad Sancte Fredman lär.\par
\vspace{10pt}
{\footnotesize\textit{Text \& Musik: Carl Michael Bellman}}
