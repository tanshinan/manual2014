\fvisa{LINGONBEN}{Bluff och Spark och Tork och Kvark}
\vspace{10pt}
Bluff och Spark och Tork och Kvark\\
voro sex små dvärgar.\\
En var ful och en var glad \\
och en var dum i huv'et.\\
Hej, sa Tork till lille Kvark,\\
känner du igelkotten Pilt?\\
Han som har varit i Paris?\\
Ja, det gjorde Ivar.\\
Hör du hans lilla runda tass\\
när som han trippar på sitt pass:\\
Tripp och trapp och trypa.\\
Se hans lilla piga!\\
\\
Tomtefar i skogens brus\\
sitter som ett päron.\\
Han har inget eget hus \\
allt i sin stora näsa.\\
Söt och blöt är sagans fé.\\
Trollen är bjudna hit på te.\\
Det lilla trollet! Pass för det!\\
Nu ska mormor bada.\\
Väva och spinna natten lång.\\
Prinsen är här i fjorton språng.\\
Hopp och hipp och huppla.\\
Hästen heter Sverker!
\newpage
Stora slottet Drummeldimp\\
ligger bortom fjärran.\\
Dit får ingen komma in\\
som ej kan baka struvor.\\
Gyllenkrull och sockertipp.\\
Kom ska vi dansa häxan våt!\\
Vill du mig här, så har du nåt.\\
Sov du lilla tryne.\\
Kungen är full av stock och sten.\\
Skogen är full av lingonben.\\
Pär är full av tomtar.\\
Hur ska Lillan orka?\\
\\
{\footnotesize\textit{Text \& Musik: Povel Ramel, 1957\\ \\ En
    ``struva'' är ett flottyrkokt bakverk}}
