\fvisa{Majsång}{Sköna maj}
\vspace{10pt}
Sköna maj, välkommen\\
till vår bygd igen! \\
Sköna maj, välkommen,\\
våra lekars vän!\\
Känslans gudaflamma\\
väcktes vid din ljusning;\\
jord och skyar stamma\\
kärlek och förtjusning;\\ 
sorgen flyr för våren,\\
glädje ler ur tåren,\\
morgonrodnad ur bekymrens moln.\par
\vspace{10pt}
Blomman låg förkolnad\\
under frost och snö;\\
höstens bleka vålnad,\\
gick hon nöjd att dö.\\
Vintern, härjarns like,\\ 
som föröder nejden\\
och i skövlat rike\\
tronar efter fejden,\\ 
satt med isad glaven\\
segrande på graven,\\
dyster själv och mörk och kall som den.\par
\newpage
Inga strålar sänktes\\
på vår morgon ner,\\
ingen daggtår skänktes\\
nordens afton mer,\\
tills, av svaner dragen,\\
maj med blomsterhatten\\
göt sitt guld i dagen,\\
purpurklädde natten,\\
vinterns spira bräckte\\ 
och ur lossat häkte\\
kallade den väna Flora fram.\par
\vspace{10pt}
Nu ur lundens sköte\\
och ur blommans knopp\\ 
stiga dig till möte\\
glada offer opp.\\
Blott ditt lov de susa,\\
dessa rosenhäckar,\\
till din ära brusa\\
våra silverbäckar,\\
och med tacksam tunga\\
tusen fåglar sjunga\\
liksom vi: Välkommen, sköna maj!
\par
\vspace{10pt}
{\footnotesize\textit{Text: Johan Ludvig Runeberg\\ Musik: Lars Magnus Béen}}
