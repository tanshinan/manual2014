\fvisa{ROSA PÅ BAL}{Tänk att jag dansar med Andersson}
\vspace{10pt}
Tänk att jag dansar med Andersson,\\
lilla jag, lilla jag, med Fritiof Andersson. \\
Tänk att bli uppbjuden av en sån \\
populär person.\par
\vspace{10pt}
Tänk vilket underbart liv det ni för.\\
Säj mej hur känns det att vara charmör,\\
sjöman och cowboy, musiker,\\
artist, det kan väl aldrig bli trist?\par
\vspace{10pt}
Nej, aldrig trist, fröken Rosa,\\
har man som er kavaljer.\\
Vart jag än ställer min kosa,\\
aldrig förglömmer jag er.\par
\vspace{10pt}
Ni är en sångmö från Helikons berg,\\
o, fröken Rosa, er linje, er färg,\\
skuldran, profilen med lockarnas krans,\\
ögonens varma glans.\par
\vspace{10pt}
Tänk, inspirera herr Andersson,\\
lilla jag, inspirera Fritiof Andersson.\\
Får jag kanhända min egen sång,\\
lilla jag, en gång?\par
\vspace{10pt}
'Rosa på bal', vackert namn eller hur?\\
början i moll och finalen i dur.\\
När blir den färdig, herr Andersson, säj,\\
visan ni diktar till mej?\par
\vspace{10pt}
Visan om er, fröken Rosa,\\
får ni i kväll till ert bord.\\
Medan vi talar på prosa,\\
diktar jag rimmande ord.\par
\vspace{10pt}
Tyst, ingen såg att jag kysste er kind,\\
känn hur det doftar från parken av lind. \\
Blommande lindar kring månbelyst stig,\\
Rosa jag älskar dig.
\par
\vspace{10pt}
{\footnotesize\textit{Text \& Musik: Evert Taube}}
