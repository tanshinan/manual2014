\fvisa{HÄRJAREVISAN}{Liksom våra fäder, vikingarna i Norden}
{\footnotesize\textit{Melodi: Gärdebylåten}}\par
\vspace{10pt}
Liksom våra fäder, vikingarna i Norden.\\
Drar vi riket runt och super oss under borden.\\
Brännvinet har blitt ett elixir,\\
för kropp såväl som själ.\\
Känner du dig liten och ynklig på jorden,\\
växer du med supen och blir stor ut i orden.\\
Slå dig för ditt håriga bröst och bli \\
en man från hår till häl.\par
\vspace{10pt}
Ja, nu skall vi ut och härja,\\
supa och slåss och svärja,\\
bränna röda stugor, slå små barn och säga fula ord.\\
Med blod skall vi stäppen färga.\\
Nu äntligen lär ja'\\
kunna dra någon riktig nytta\\
av min Hermodskurs i mord.\par
\vspace{10pt}
Hurra, nu ska man äntligen få röra benen,\\
hela stammen jublar och det spritter i grenen.\\
Tänk att än en gång få spränga fram\\
på Brunte i galopp!\\
Din doft o kära Brunte är trots sin brist i hygienen,\\
för en vild mongol minst lika ljuv som syrenen.\\
Tänk att på din rygg få rida runt\\
i stan och spela topp.\par
\vspace{10pt}
Ja, nu skall vi ut och härja...\par
\vspace{10pt}
Ja, mordbränder är klämmiga ta fram fotogenen\\
och eftersläckningen tillhör just de fenomenen\\
inom brandmansyrket som jag tycker\\
det är nån nytta med.\\
Jag målar för mitt inre upp den härliga scenen;\\
Blodrött mitt i brandgult, ej ens prins Eugen en\\
lika mustig vy kan måla, \\
ens om han målade med sked.\par
\vspace{10pt}
Ja, nu skall vi ut och härja...\par
\vspace{10pt}
{\footnotesize\textit{Text: Hans Alfredsson och Levin}}\par
\vspace{10pt}
{\footnotesize\textit{Ur Lundaspexet Djingis Kahn 1954, första versen ej ur originalet.}}\par
\vspace{10pt}
\index{DATAVETARNAS HÄRJARVISA}
Nu ska vi kompilera,\\
supa och penetrera\\
söka rekusioner ända in\\
till fixpunktssemantik.\\
Av induktion och basfall\\
blir man med lätthet asknall.\\
Hashtabeller parsas rekursivt\\
ifrån hår till häl\par
\vspace{10pt}
{\footnotesize\textit{Text: Pär Mattsson och Magnus Ingelbo från DVL
    Uppsala.}}
