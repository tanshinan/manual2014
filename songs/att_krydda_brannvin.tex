\tvisa{ATT KRYDDA BRÄNNVIN}
\vspace{10pt}
\hspace{10pt}När man gör egna kryddningar bör det brännvin 
man använder helst inte vara för starkt om kryddsmaken 
skall komma till sin fulla rätt. Man utgår från okryddat 
brännvin, renat eller vodka.\\
\hspace*{10pt}Oftast går tillverkningen av essens till så att man 
lägger bär eller blommor i en flaska och häller över så 
mycket brännvin att det täcker. Sedan låter man flaskan 
stå i rumstemperatur en vecka och silar därefter bort bär 
eller blommor. Är essensen grumlig ska man låta den stå 
upp till ett halvår. Sedan häller man försiktigt över 
essensen till en annan flaska så att bottensatsen blir kvar.
\par\vspace{10pt}
\underline{Björkskottens brännvin}
\par\vspace{10pt}
\hspace*{10pt}Låter det inte romantiskt? Och björkar har vi ju gott 
om så vi behöver inte vara rädda för att göra någon större 
skada i naturen.\\
\hspace*{10pt}Strax innan björken fått sina första ``musöron'', när 
knopparna fått en liten, liten tipp av grönt, då är den rätta 
tiden att plocka skotten. Stoppa dem i en flaska, häll 
brännvinet över och låt essensen stå och dra i ungefär en 
vecka i rumstemperatur.\\
\hspace*{10pt}Då har ni fått en guldbrun vätska som smakar 
ganska starkt. Blanda ett drygt snapsglas essens med en 
halv flaska brännbin. Ingen kan gissa hur drycken är 
kryddad - men god är den!
\newpage
\underline{Besk - Malörtsbrännvin}
\par\vspace{10pt}
\hspace*{10pt}Besk är en gammal pålitlig snaps där grundkryddan 
är malört. Se bara upp när ni plockar malört så att ni inte 
förväxlar den med den vanliga svinmålan! Malörten har 
silvergrå, lite ludna blad och en hård stam, som är lätt att 
bryta av. Blommorna är små och oansenliga och gula.\\
\hspace*{10pt}Vill man göra ett riktigt fint malörtsbrännvin skall 
man bara använda blommorna. Först ska de hänga och 
torka på sina stjälkar, helst i ett dragigt rum, sedan är det 
lätt att dra av blommorna.\\
\hspace*{10pt}Lägg dem på ett fat, slå över en smula brännvin och 
tänd på. Elden slocknar snart och då stoppas blommorna i 
en flaska som sedan fylls med brännvin. Får stå en vecka, 
varefter essensen silas och lagras.\\
\hspace*{10pt}Det går också att använda både blad och de finare 
stjälkarna till essensen om man så vill.\\
\hspace*{10pt}Många tycker särskilt mycket om att få en ``besk'' 
till julens feta mat. Av gammalt har besken namn om sig 
att hjälpa till med matsmältningen.
\par\vspace{10pt}
{\footnotesize\textit{Ur boken ``Snapsvisor och brännvinsrecept''.}}
