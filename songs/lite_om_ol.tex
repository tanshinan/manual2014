\visa{LITE OM ÖL}
\vspace{10pt}
Öl, alkoholdryck framställd av mältad säd (oftast korn), humle, vatten och jäst.

Konsten att brygga öl utvecklades san under yngre stenåldern. I Mesopotamien finns uppgifter om flera olika sorters öl redan från ca 3000 f.Kr., och i Egypten framställdes öl åtminstone från Gamla rikets tid (ca 2700–2270 f.Kr.).
I Norden har påträffats gravkärl från bronsåldern (ca 1800–500 f.Kr.) med rester av intorkat öl.
Metoderna har växlat och kan variera än idag. Man utgår dock alltid från en sädesråvara som mältas, för att dess stärkelse under den följande vörtbryggningen skall kunna lösas i vatten och omvandlas till socker, vilket i sin tur jäses till alkohol, kolsyra och smakämnen. Det så erhållna ölet smaksätts normalt med någon krydda, numera praktiskt taget alltid humle.
Öl med en alkoholhalt över 0,5 volymprocent beskattas efter alkoholhalten. Skatten är 1,66 krona per volymprocent alkohol och liter öl. Ett särskilt undantag görs för öl med en alkoholhalt under 2,8 volymprocent, där skatten är 0 kronor.
