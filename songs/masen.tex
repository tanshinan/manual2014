\fvisa{Måsen}{Det satt en mås på en klyvarbom}
{\footnotesize\textit{Melodi: Månvisa}}\par
\vspace{10pt}
Det satt en mås på en klyvarbom\\
och tom i krävan var kräket.\\
Tungan lådde vid skepparns gom,\\
där han satt uti bleket.\\
Jag vill ha sill hördes måsen rope\\
och skepparn svarte: Jag vill ha OP\\
Om blott jag får, om blott jag får.\par
\vspace{10pt}
Nu lyfter måsen från klyvarbom\\
och vinden spelar i tågen.\\
OPn svalkat har skepparns gom,\\
jag önskar blott att jag såg en.\\
Så nöjd och lycklig den arme saten,\\
han sätter storsegel den krabaten.\\
Till havs han far och halvan tar.\par
\vspace{10pt}
Nu månen vandrar sin tysta ban\\
och tittar in genom rutan.\\
Då tänker jag att på ljusan dag\\
då kan jag klara mig utan.\\
Då kan jag klara mig utan måne,\\ 
men utan renat och utan skåne,\\
det vete fan, det vete fan.\par
\newpage
Den mås som satt på en klyvarbom,\\
den är nu död och begraven,\\
och skepparn som drack en flaska rom,\\
han har nu drunknat i haven.\\
Så kan det gå om man fått för mycket,\\
om man för brännvin har fattat tycke.\\
Vi som har kvar, vi resten tar.\par
\vspace{15pt}
\textbf{Moosen}\\
Det satt en älg i en klyvartopp,\\
förklädd i älgjaktens månad.\\
Han var befjädrad till horn och kropp\\
ja, skepparen blev rätt förvånad\\
``jag är en mås, goa skepparn'' ljög den\\
förklädda älgen. Därefter flög den.\\
Mjukt föll den sen, på skepparen.\par
\vspace{15pt}
\textbf{Musen}\\
Det satt en mus i en hushållsost\\
och åt och åt utan måtta\\
tills osten blev till en mushåls-ost\\
och han en klotformad råtta.\\
``Så bra'' sa musen ``att va' en fettboll\\
nu kan jag rulla med hast åt rätt håll:\\
Ostindien, Ostindien''
