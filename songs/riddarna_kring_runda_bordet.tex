\fvisa{Riddarna kring runda bordet}{Jag reste med ett tåg någonstans}
\vspace{10pt}
Jag reste med ett tåg någonstans;\\
Vägen var lång och innan vi kom fram\\
så föll jag i sömn och drömde en dröm\\
om tider och vänner jag trodde jag glömt.\par
\vspace{10pt}
Jag kom till min första lägenhet,\\
där luften var unken och klibbig och het.\\
Där stank gamla mattor, där stank stearin;\\
Men lukten var helig, för lukten var min.\par
\vspace{10pt}
Jag satt runt ett bord, täckt av flaskor och glas,\\
med alla dom andra i mitt blodsbrödralag.\\
Det var vännerna jag älskade, dom enda jag haft\\
som gett mej visioner och oskuldsfull kraft.\par
\vspace{10pt}
Därute var Världen självisk och ful,\\
men moralen var hög runt vårt flämtande ljus;\\
En framgång för någon var allas triumf,\\
och ingen stod ensam i motgångens stund.\\
Vi var en för alla, vi var alla för en;\\
Och vår tanke var enkel och självklar och ren;\\
Vi svor att bekämpa det vi visste var fel;\par
\vspace{10pt}
För vi anade inget om framtidens spel.\par
\vspace{10pt}
Vi talade som om tiden står still,\\
som om livet beter sej precis som man vill.\\
Som om Kosmos får plats i ett ungkarlsrum,\\
som om Världen är bordet om bordet är runt.\par
\vspace{10pt}
Men tiden förändras och bröder med den.\\
Och vi reste ju trots allt mot framtiden.\\
Men vi lovade att strida på var sina håll;\\
Fast chansen att lyckas var mindre än noll.\par
\vspace{10pt}
En del har försvunnit, en del har jag glömt;\\
En del har jag träffat, en del har jag fördömt.\\
En del lär förneka att dom varit min vän;\\
Dom ger jag min kärlek, för dom behöver den än.\par
\vspace{10pt}
Jag önskar, jag önskar, jag önskar ibland\\
att jag åter kunde knyta ett blodsbrödraband.\\
En dryg million mitt på bordet, min vän,\\
det skulle jag betala för att få göra det igen!\par
\vspace{10pt}
{\footnotesize\textit{Text: Björn Afzelius}}
