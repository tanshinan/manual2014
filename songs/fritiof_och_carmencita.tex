\fvisa{FRITIOF OCH CARMENCITA}{Samborombon en liten by förutan gata}
\vspace{10pt}
Samborombon en liten by förutan gata,\\
den ligger inte långt från Rio de la Plata.\\
Nästan i kanten av den blåa Atlanten\\
och med pampas bakom sig många hundra gröna mil,\\
dit kom jag ridande en afton i april,\\
för jag ville dansa tango.\\
\\
Dragspel, fiol och mandolin\\
hördes från krogen och i salen steg jag in.\\
Där på bänken i mantilj och med en ros vid sin barm\\
satt den bedårande lilla Carmencita\\
\\
Mamman, värdinnan satt i vrån\\
hon tog mitt ridspö, min pistol och min manton\\
jag bjöd upp och Carmencita sa: si gracias, senor\\
vamos a bailar - este tango\\
\\
Carmencita lilla vän, håller du utav mig än?\\
får jag tala med din pappa och din mamma\\
jag vill gifta mig med dig, Carmencita.\\
\\
Nej, don Fritiof Andersson,\\
kom ej till samborombom\\
om ni hyser andra planer när det gäller mig\\
än att dansa tango\\
\\
Ack Carmencita, gör mej inte så besviken,\\
jag tänkte skaffa mej ett jobb här i butiken,\\
sköta mej noga, bara spara och knoga,\\
inte spela och dricka men bara älska dej,\\
säj, Carmencita, det är ändå blott med mej,\\
säj, som du vill dansa tango?\\
\\
Nej Fritiof, ni förstår musik, men jag tror inte\\
ni kan stå i en butik och förresten sa min pappa just idag\\
att han visste vem som snart skulle fria till hans dotter.\\
En som har tjugotusen kor\\
och en estancia som är förfärligt stor.\\
Han har prisbelönta tjurar, han har oxar, får och svin,\\
och han dansar underbart tango.\\
\\
Carmencita, lilla vän, akta dej för rika män!\\
Lyckan den bor ej i oxar eller kor, och den kan\\
heller inte köpas för pengar.\\
Men min kärlek gör dej rik, skaffa mej ett jobb i er butik!\\
Och när vi blir gifta söta ungar ska du få,\\
som kan dansa tango!
\par
\vspace{10pt}
{\footnotesize\textit{Text \& Musik: Evert Taube}}
