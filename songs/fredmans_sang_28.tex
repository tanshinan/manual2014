\fvisa{Fredmans Sång N:o 28}{Movitz skulle bli Student}
\vspace{10pt}
Movitz skulle bli Student;\\
Han Upsala betrakta,\\
Börja mumla excellent\\
Grammatica contracta;\\
	Dum och tjock,\\
	Hic hæc hoc\\
	Han sig genast lärde,\\
Hyrde sig en svarter rock,\\
Kyronii öl förtärde.\\
\\
Där satt han som misanthrop,\\
Men röder som en vallmo,\\
Vid sin stånka och sit stop,\\
Och conjugera Amo;\\
	Hur han drack,\\
	Ölet stack,\\
	Kärlek hjärnen brydde;\\
Movitz tog sitt pick och pack,\\
Och lärdoms sätet flydde.\\
\\
Med en vredgad min han tog\\
Båd Puffendorff och Grotius,\\
Och dem bus i väggen slog,\\
Så bister som Stygotius,\\
	Sjöng hurra,\\
	Skrek Verda,\\
	Och åt Krögarn panta\\
Lexicon, Colloquia,\\
Och Zopfens varianta.\\
\\
Tre Studenter, certim tre,\\
Til Stockholm sig utstyra,\\
Stadna på Tre Remmare,\\
I Skrubben Numro Fyra;\\
	Movitz stolt,\\
	I sin kolt,\\
	Satt sig där som Præses,\\
Drack så bålt, så det var bålt,\\
Och utgaf nya Theses.\\
\\
Första Thesis blef nu den:\\
Om med moraliteten\\
Enligt är för Bacchi män\\
At ändra om dieten.\\
	Pro och Pro,\\
	Contra, jo,\\
	Nej och Ja nu skalla;\\
Movitz ropte Posito,\\
Och Posito skrek alla.\\
\\
Andra Thesis blef den här:\\
Hvad skilnad sig besticker\\
Mellan Öl pluraliter\\
Och en Person som dricker;\\
	Ratio? jo,\\
	Dubito,\\
	Skrek en och orera.\\
Movitz ropte Habeo,\\
Och Kyparn slog i mera.\\
\\
Tredje Thesis skulle ges,\\
Men Præses damp af stolen,\\
Och en Opponent så hes\\
Damp med i Capriolen.\\ 
	Fredman kom,\\
	Filibom!\\
	Med musik och flickor;\\
Och de lärde vände om\\
Som åsnor och borickor.
