\visa{O, GAMLA KLANG- OCH JUBELTID}
{\footnotesize\textit{Melodi: Oh alte Burschenherrlichkeit}}\par
\vspace{10pt}
O gamla klang och jubeltid\\
ditt minne skall förbliva\\
och än åt livets bistra strid,\\
ett rosigt skimmer giva.\\
Snart tystnar allt vårt yra skämt,\\
vår sång blir stum, vårt glam förstämt.\\
O, jerum, jerum, jerum.\\
O, quae mutatio rerum!\par
\vspace{10pt}
Var äro de som kunde allt,\\
blott ej sin ära svika.\\
Som voro män av äkta halt\\
och världens herrar lika?\\
De drogo bort från vin och sång\\
till vardagslivets tråk och tvång.\\
O, jerum, jerum, jerum.\\
O, quae mutatio rerum!\\
\begin{leftborder}
\begin{tabular}{l l}
  (Filosofer)  & Den ene vetenskap och vett\\		
               & in i scholares mänger,\\
  (Jurister)   & Den andre i sitt anlets svett\\
               & på paragrafer vränger,\\
  (Teologer)   & en plåstrar själen, som är skral,\\
  (Medicinare) & en lappar hop dess trasiga fodral;\\
  (Alla)       & O, jerum, jerum, jerum,\\
               & O, quae mutatio rerum!
\end{tabular}
\end{leftborder}
\newpage
Men hjärtat i en sann student,\\
kan ingen tid förfrysa.\\
Den glädjeeld, som där har tänt,\\
hans hela liv skall lysa.\\
Det gamla skalet brustit har\\
men \textit{kärnan} finnes frisk dock kvar,\\
och vad han än må mista,\\
den skall dock aldrig brista!\par
\vspace{10pt}
Så sluten, bröder, fast vår krets,\\
till glädjens värn och ära!\\
Trots allt vi tryggt och väl tillfreds,\\
vår vänskap trohet svära.\\
Lyft bägar'n högt, och klinga vän!\\
De gamla gudar leva än\\
\revrpt bland skålar och pokaler\rpt\par
\vspace{10pt}
{\footnotesize\textit{Text: August Lindh, 1925\\ Musik:
Eugen Höfling, 1825}}\par
\vspace{10pt}
{\footnotesize\textit{Vid ``kärnan' dunkas näven i bordet en
gång. Den sista versen sjungs stående samt stående på stolen, efter
sista versen sjungits sätter sig ingen åter till bords.}}
