\fvisa{Smedsvisa}{En gång i min ungdom älskade jag}
\vspace{10pt}
En gång i min ungdom älskade jag\\
en flicka med rena och sköna behag.\\
Hon lovfte mig tro, i lust och i nöd\\
allt in till sin blekaste död.\\
Hej hopp faderi faderalladerej\\
Hej hopp faderi faderalladerej\\
Hon lovfte mig tro, i lust och i nöd\\
allt in i sin blekaste död.\\
\\
Hon var som en lilja vit uti hyn\\
den vackraste kvinna man skådat i byn,\\
ett smittande skratt, en lustiger sång\\
vi älskade sommaren lång.\\
Hej hopp faderi faderalladerej\\
Hej hopp faderi faderalladerej\\
ett smittande skratt, en lustiger sång\\
vi älskade sommaren lång\\
\\
Men kärleken vissna', kärleken dog\\
vid Mikaels mässa den flickan fått nog.\\
Hon fann sig en riker och högfärdig man\\
sa ``Tack och adjö'' och försvann.\\
Hej hopp faderi faderalladerej\\
Hej hopp faderi faderalladerej\\
Hon fann sig en riker högfärdig man\\
sa ``Tack och adjö'' och försvann.\\
\\
Nu står jag vid städet sliten och grå\\
och hammaren bultar och hjärtat likså.\\
Den flickan hon kommer aldrig igen,\\
hon är hos sin nyfunne vän.\\
Hej hopp faderi faderalladerej\\
Hej hopp faderi faderalladerej\\
Den flickan hon kommer aldrig igen\\
men sången den trallar jag än.\\
Tra raj . . .
\vspace{15pt}
\fvisa{Smedsvisa den korta}{En gång i min ungdom älskade jag}
\vspace{10pt}
En gång i min ungdom älskade jag\\
sa ``Tack och adjö'' och försvann.
