\visa{PUNSCHHISTORIK}
\vspace{10pt}
Punschen är en likörliknande spritdryck som beredes av arrak (eller rom), sprit, socker och vatten samt smaksättes med bland annat cognac, rom eller whisky och eventuellt citronsaft. Namnet Punsch (eng. punch, ursprungligen av sanskrit panca, fem, efter de ursprungliga fem beståndsdelarna).\\
Den användes i Sverige första gången sannolikt 1733 vid festligheter i anledning av hemkomsten av ostindiefararen Friederi'cus Rew Sue'ciae, vilkens besättning i Kina lärt känna drycken och dess beredning. Till en början förtärdes Punschen alltid varm, färdigberedd vid bordet, men mot mitten av 1840-talet, sedan lagring (minst 4 mån) tillkommit, vanligen kall.\\
Punschens kvalitet är i främsta hand beroende av arrakens beskaffenhet samt proportionerna mellan de olika ingredienserna. Den mest framstående punschen innehåller dock en relativt stor mängd arrak.\\
Den svenska fabriksmässiga punschtillverkningen daterar sig från 1845. Intill 1800-talets ingång var användningen av Punsch å allmogens gästabud förbjuden såsom lyx.\\
Vissa punschtillverkare hyrde lagringsplats för sin punschfat på svenska örlogsfartyg för att punschen skulle får lagras till sjös, man fick en sk "Sjörullad" punsch.\\
Det sägs att Konung Karl Gustav Folke Hubertus Bernadotte spar en skvätt punsch då han äter ärtsoppa, för att inmundiga denna senare med de tillhörande pannkakorna.
