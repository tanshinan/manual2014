\fvisa{Bordsvisa}{När skämtet tar ordet vid vänskapens bord}
\vspace{10pt}
När skämtet tar ordet vid vänskapens bord\\
med fingret åt glasen, som dofta,\\
så drick och var glad: på vår sorgliga jord\\
man gläder sig aldrig för ofta.\\
En blomma är glädjen: i dag slår hon ut,\\
i morgon förvissnar hon redan,\\
just nu, då du kan, hav en lycklig minut\\
och tänk på den kommande sedan.\par
\vspace{10pt}
Vem drog ej en suck över tidernas lopp?\\
Dock sitt ej och dröm på kalaset!\\
Här lev i sekunden, och hela ditt hopp\\
se fyllas och tömmas - i glaset!\\
Här sörj ej för glaset: om fullt, så drick ut;\\
om tomt, så försänd det att fyllas;\\
och minns, att det sköna och goda förut,\\
sen glädjen och nöjet, må hyllas.\par
\newpage
Ty ägne vi först åt värdinnan en skål.\\
Vad vore vår fröjd utan henne?\\
Sen prise vi värden och särskilt hans bål.\\
Vad vore vårt mod utan denne?\\
Dem båda förene ett glas och en sång:\\
de själva så skönt sig förente.\\
Med druvorna myrten blev skapt på en gång:\\
vem ser ej vad himmelen mente?\par
\vspace{10pt}
För övrigt må värden ge alltid nytt skäl\\
till ständig omsättning av glasen\\
och visa, att rangen är nyttig likväl -\\
till skålarnas mängd på kalasen!\\
Men förr'n han är färdig med klang och harang,\\
vi skynda att självmant dricka\\
och helga ett glas, som är över all rang,\\
i tysthet - envar åt sin flicka!\par
\vspace{10pt}
{\footnotesize\textit{Text och melodi: Frans Michael Franzén}}
