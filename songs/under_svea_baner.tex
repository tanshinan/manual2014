\visa{UNDER SVEA BANÈR}
\vspace{10pt}
Under Svea Banér Himlen seger oss ger;\\
Då för Konung och Land Äran lyfter sin hand.\\
Ännu Svearnes mod sif bereder en Stod\\
Utaf Lagrarne höjd, fast besglad mes blod.\\
Himlen gifve oss frid!\\
Men om den icke vinne utan vapen och strid,\\
Blifve Segern då vår, ögat offre sin tår\\
Den som faller, ock rätta till vår tacksamhet får.\par
\vspace{10pt}
{\footnotesize\textit{Text: Samuel Ödmann\\ Musik: Johann Christian
Friedrich Haeffner}}\par
{\footnotesize\textit{Anses vara den första upsaloensiska studentsången. Musiken hade
skrivits tidigare som soldatkör till Hæffners opera “Renaud”
(1801).}}\par
\hspace{10pt}{\footnotesize\textit{Framfördes första gången, mitt under kriget mot Ryssland,
den 24 oktober 1808. Detta skedde vid en
uppvakting för fältmarskalkten Klingspor på genomresa från Finland
till Stockholm som gästade landshövding Uppsala slott.  Uppsalas
studenter marscherade då upp till slottet. Sångarna gick i täten och
sjöng denna sång.}}
