\fvisa{KALMAREVISAN}{För uti Kalmare stad}
\vspace{10pt}
\textbf{För uti Kalmare stad}\\
ja där finns det ingen kvast\\
förrän lördagen.\\
\textbf{Hej dick}\\
Hej dack\\
\textbf{Jag slog i}\\
och vi drack\\
\textbf{Hej dickom dickom dack}\\
hej dickom dickom dack.\\
För uti Kalmare stad\\
ja där finns det ingen kvast\\
förrän lördagen.\\
\\
\revrpt \textbf{När som bonden kommer hem}\\
kommer bondekvinnan ut\rpt\\
och är stor i sin trut\\
\textbf{Hej dick...}\\
\\
\revrpt \textbf{Var är pengarna du fått?}\\
Jo, dom har jag supit opp!\rpt\\
Uppå Kalmare slott.\\
Hej dick...\\
\\
\revrpt \textbf{Jag skall mäla dig an}\\
för vår kronbefallningsman\rpt\\
Och du skall få skam\\
\textbf{Hej dick...}
\newpage
\revrpt \textbf{Kronbefallningsmannen vår}\\
satt på krogen i går\rpt\\
Och var full som ett får.\\
\textbf{Hej dick...}\\
\\
{\footnotesize\textit{Text: G S Kallstenius}}\\
\\
(\textit{Tre små tillägg})\\
\revrpt \textbf{Va' sa' bonnen ha te' mat?}\\
Sura sillar och potat\rpt\\
det blir sillsallat.\\
\textbf{Hej dick...}\\
\\
\revrpt \textbf{Säg var är din lab-rapport}\\
Jo, den har jag supit bort!\rpt\\
För den var så kort!\\
\textbf{Hej dick...}\\
\\
\revrpt\textbf{Ordföranden vår}\\
var på 100-kör igår.\rpt\\
Dom bar ut han på bår!\\
\textbf{Hej dick...}\\
\\
{\footnotesize\textit{I Linköping är det inget ``Hej dick...''-ande på
    första versen, dock reprisande som följande verser.}}
