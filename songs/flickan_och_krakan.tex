\fvisa{FLICKAN OCH KRÅKAN}{Jag satt häromdagen och läste min tidning}
\vspace{10pt}
Jag satt häromdagen och läste min tidning\\
en dag som så många förut.\\
O jag tänkte på alla dom drömmar man drömt som\\
en efter en har tatt slut\par
\vspace{10pt}
Då såg jag en bild av en flicka\\
med en skadskjuten kråka i famn\\
hon springer iväg genom skogen\\
så fort som hon någonsin kan\par
\vspace{10pt}
Och hon springer med fladdrande lockar\\
hon springer på taniga ben\\
o hon bönar och ber och hon hoppas och tror\\
att det inte ska vara för sent\par
\vspace{10pt}
Flickan är liten och hennes hår är så ljust\\
o hennes kind är så flämtande röd\\
kråkan är klumpig och kraxande svart\\
om en stund är den alldeles död\par
\vspace{10pt}
Men flickan, hon springer för livet\\
hos en skadskjuten fågel i famn\\
hon springer mot trygghet och värme\\
för det som är riktigt och sant\par
\vspace{10pt}
O hon springer med tindrande ögon\\
hon springer på taniga ben\\
för hon vet att det är sant, det som pappa har sagt\\
att finns det liv är det aldrig för sent\par
\vspace{10pt}
O jag började darra i vånda och nöd\\
jag skakade av rädsla och skräck\\
för jag visste ju alldeles tydligt och klart\\
att det var bilden av mig som jag sett\par
\vspace{10pt}
För mitt hopp är en skadsjuten kråka\\
och jag är ett springande barn\\
som tror det finns någon som kan hjälpa mig än\\
som tror det finns nån som har svar\par
\vspace{10pt}
O jag springer med bultande hjärta\\
jag springer på taniga ben\\
O jag bönar och ber, fast jag egentligen vet\\
att det redan är alldeles för sent
\par
\vspace{10pt}
{\footnotesize\textit{Text \& Musik Mikael Wiehe}}
