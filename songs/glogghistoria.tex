\tvisa{Glögghistoria}
\vspace{10pt}
\noindent\hspace*{10pt}Vi har druckit varm, söt julglögg av vin och kryddor sedan slutet av
1800-talet. Däremot har vi efter utländskt mönster värmt oss med
glödgat vin sedan 1600-talet. Medan seden att krydda vin säkerligen är
lika gammal som vinet självt. Antagligen började det med att dölja
dålig smak på vinet, men så småningom blev det också ett sätt att
skryta med sina dyra kryddor.\par
\noindent\hspace*{10pt}Den närmaste föregångaren till julglögg är annars den franska drucken blûlot. Den görs av cognac som värms upp i en gryta. Sedan täner man elden på spriten och öser den över en sockertopp, placerad på ett galler över grytan.\par
\noindent\hspace*{10pt}Färdiggjord glögg på flaska kom strax efter sekelskiftet. De flesta privata vinhandlare höll med egna varianter. Alla med fantasifulla etiketter.\par
\vspace{10pt}
{\footnotesize\textit{Ur Systembolagets tidning ``Julnytt - om glögg och julöl 1995''}}
