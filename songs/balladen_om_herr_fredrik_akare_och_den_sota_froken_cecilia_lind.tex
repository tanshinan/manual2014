\svisa{BALLADEN OM HERR FREDRIK ÅKARE}{Från Öckerö loge hörs dragspel och bas}{BALLADEN OM HERR FREDRIK ÅKARE\\OCH DEN SÖTA FRÖKEN CECILIA LIND}
\vspace{8pt}
Från Öckerö loge hörs dragspel och bas\\
och fullmånen lyser som var den av glas.\\
Där dansar Fredrik Åkare kind emot kind\\
med lilla fröken Cecilia Lind.\par
\vspace{6pt}
Hon dansar och blundar så nära intill,\\
hon följer i dansen precis vart han vill.\\
Han för och hon följer så lätt som en vind,\\
Men säg varför rodnar Cecilia Lind?\par
\vspace{6pt}
Säg var det för det Fredrik Åkare sa:\\
Du doftar så gott och du dansar så bra.\\
Din midja är smal och barmen är trind.\\
Vad du är vacker, Cecilia Lind.\par
\vspace{6pt}
Men dansen tog slut och vart skulle dom gå?\\
Dom bodde så nära varandra ändå.\\
Till slut kom dom fram till Cecilias grind.\\
Nu vill jag bli kysst, sa Cecilia Lind.\par
\vspace{6pt}
Vet hut, Fredrik Åkare, skäms gamla karln!\\
Cecilia Lind är ju bara ett barn.\\
Ren som en blomma, skygg som en hind.\\
Jag fyller snart sjutton, sa Cecilia Lind.\par
\vspace{6pt}
Och stjärnorna vandra och timmarna fly\\
och Fredrik är gammal men månen är ny.\\
Ja, Fredrik är gammal men kärlek är blind.\\
Åh, kyss mig igen, sa Cecilia Lind.\par
\vspace{8pt}
{\footnotesize\textit{Text: Cornelis Vreeswijk}}
