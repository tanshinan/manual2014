\fvisa{Roma}{Jag satt en kväll i Roma}
{\footnotesize\textit{Melodi: Manschettvisan}}\par
\vspace{10pt}
Jag satt en kväll i Roma\\
och njöt cigarr aroma,\\
mitt fagra fysionoma\\
mot mången kvinna le.\\
\\
Då kom en brallisita\\
med former trés bonita.\\
Ej utan all invita\\
var hennes bystament.\\
\\
Hon sa på Roma måle:\\
''Ni vara fagro tvåle:\\
Jag har förslag frivole.\\
Vi älska med varann´´.\\
\\
Mitt inne bland spaljene\\
hon visa molto bene.\\
Mitt inne ibland grene\\
vi hade téte à téte.\\
\\
Vi téte lite granda,\\
jag göra observanda\\
att kvinna confirmanda\\
hon vara långt ifrån.\\
\\
Jag klänga kors och tvärso.\\
Det var en riktig pärzo.\\
Till sist bli kall om stjärtzo\\
och vi gå hem till hon.\\
\\
Jag kysste'na mitt i nia.\\
Hon ropa: ''Madre mia!\\
Ni vara kyssgeni, ja´´.\\
Jag släckte armatur.\\
\\
Då kom där en señore\\
och skrek i hög tenore:\\
''Ni kan be Fadre Våre,\\
ty jag är hennes man´´.\\
\\
Han gjorde sig beretto\\
och drog en jädrans lång stiletto.\\
Jag kavla upp manschetto,\\
drog fram min nagelfil.\\
\\
Så började duello\\
om denna bonne mamsello.\\
I hans akterkastello\\
jag rände mitt instrument.\\
\\
Han vred sig i spirale\\
och ropade: ''Tvi vale´´,\\
och la sig horisontale\\
och sluta sina dar.\\
\\
Nu ligger han i jordo,\\
begravd av fattigvårdo,\\
och jag blev fast för mordo\\
och slängd i finkament.\\
\\
Nu sitter jag i cello\\
och fäller många mjällo\\
för denna bonne mamsello,\\
ty hon hörs aldrig av.
