\fvisa{Fredmans Epistel N:\lowercase{o} 81}{Märk, hur' vår skugga, märk, Movitz mon Frère}
{\footnotesize\textit{Till grälmakar Löfberg i stärbhuset vid Dantobommen, diktad vid graven.}}\par
\vspace{10pt}
Märk, hur' vår skugga, märk, Movitz mon Frère,\\
inom ett mörker sig slutar,\\
hur guld och purpur i skoveln, den där,\\
byts till grus och klutar.\\
Vinkar Charon från sin brusande älv,\\
och tre gånger sen dödgrävaren själv,\\
mer du din druva ej kryster.\\
Därföre Movitz kom hjälp mig och välv\\
gravsten över vår syster!\par
\vspace{10pt}
Ack längtansvärda och bortskymda skjul\\
under de susande grenar,\\
där tid och döden en skönhet och ful\\
till ett stoft förenar!\\
Till dig aldrig avund sökt någon stig;\\
lyckan, eljest uti flykten så vig,\\
aldrig kring grifterna ilar.\\
Ovän där väpnad, vad synes väl dig?\\
bryter fromt sina pilar.\par
\newpage
Lillklockan klämtar till storklockans dön,\\
lövad står kantorn i porten\\
och vid de skrålande gossarnas bön\\
helgar denna orten.\\
Vägen opp till templets griftprydda stad\\
trampas mellan rosors gulnade blad,\\
multnade plankor och bårar;\\
till dess den långa och svartklädda rad,\\
djupt sig buga med tårar.\par
\vspace{10pt}
Så gick till vila, från slagsmål och bal,\\
grälmakar Löfberg, din maka,\\
där, dit åt gräset långhalsig och smal\\
du än glor tilbaka.\\
Hon från Dantobommen skildes i dag,\\
och med henne alla lustiga lag.\\
vem skall nu fl askan befalla?\\
Torstig var hon och uttorstig är jag;\\
vi ä torstiga alla.\par
\vspace{10pt}
{\footnotesize\textit{Text \& Musik: Carl Michael Bellman}}
