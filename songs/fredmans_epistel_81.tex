\fvisa{Fredmans Epistel N:o 81}{Märk hur' vår skugga}
\vspace{10pt}
Märk hur' vår skugga, märk Movitz Mon Frere!\\
Innom et mörker sig slutar,\\
Hur Guld och Purpur i Skåfveln, den där,\\
Byts til grus och klutar.\\
Vinkar Charon från sin brusande älf,\\
Och tre gånger sen Dödgräfvaren sjelf,\\
Mer du din drufva ej kryster.\\
Därföre Movitz kom hjelp mig och hvälf\\
Grafsten öfver vår Syster.\\
\\
Ach längtansvärda och bortskymda skjul,\\
Under de susande grenar,\\
Där Tid och Döden en skönhet och ful\\
Til et stoft förenar!\\
Til dig aldrig Afund sökt någon stig,\\
Lyckan, eljest uti flygten så vig,\\
Aldrig kring Grifterna ilar.\\
Ovän där väpnad, hvad synes väl dig?\\
Bryter fromt sina pilar.\\
\\
Lillklockan klämtar til Storklockans dön,\\
Löfvad står Cantorn i porten;\\
Och vid de skrålande Gåssarnas bön,\\
Helgar denna orten.\\
Vägen opp til Templets griftprydda stad\\
Trampas mellan Rosors gulnade blad,\\
Multnade Plankor och Bårar;\\
Til dess den långa och svartklädda rad,\\
Djupt sig bugar med tårar.\\
\\
Så gick til hvila, från Slagsmål och Bal,\\
Grälmakar Löfberg, din maka;\\
Där, dit åt gräset långhalsig och smal,\\
Du än glor tilbaka.\\
Hon från Danto bommen skildes i dag,\\
Och med Hänne alla lustiga lag;\\
Hvem skall nu Flaskan befalla.\\
Torstig var hon och uttorstig är jag;\\
Vi ä torstiga alla.
