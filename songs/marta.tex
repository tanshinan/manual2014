\fvisa{MÄRTA}{Det var en flicka, hon var en snärta}
{\footnotesize\textit{Melodi: I Apladalen i Värnamo}}\par
\vspace{10pt}
Det var en flicka, hon var en snärta\\
hon bar det klingande namnet Märta.\\
Vi bruka' träffas så där ibland,\\
tills helt mitt hjärta hon satt i brand.\par
\vspace{10pt}
En dag sa' Märta: "Ska vi spatsera\\
omkring i parken och resonera.\\
Vi kan gå runt ner på stan en stund,\\
sen får du följa mig på mitt rum."\par
\vspace{10pt}
På ett bananskal, som låg på vägen,\\
där halka Märta, jag blev förlägen,\\
ty kan ni tänka vad jag fick se?\\
Min Märta hade ett ben av trä.\par
\vspace{10pt}
Hom var en flicka av rätta sorten,\\
det fick jag se när vi kom till porten.\\
Hon hala klänningen upp en flik,\\
där hängde portnyckeln på en spik.\par
\vspace{10pt}
Och när som Märta hon hade somnat,\\
och hennes träben det hade domnat,\\
jag rista' namnet i benets bark,\\
allt medan Märta i sängen snark.\par
\vspace{10pt}
En tös med träben ska ni förvisa,\\
det har ni lärt er av denna visa.\\
Jag plocka stickor i fjorton dar,\\
och har väl alltjämt ett flertal kvar.
