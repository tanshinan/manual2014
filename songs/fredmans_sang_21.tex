\fvisa{Fredmans Sång N:o 21}{Så lunka vi så småningom}
\vspace{10pt}
Så lunka vi så småningom\\
Från Bacchi buller och tumult,\\
När döden ropar, Granne kom,\\
Ditt timglas är nu fullt.\\
Du Gubbe fäll din krycka ner,\\
Och du, du Yngling, lyd min lag,\\
Den skönsta Nymph som åt dig ler\\
Inunder armen tag.\\
Tycker du at grafven är för djup,\\
Nå välan så tag dig då en sup,\\
Tag dig sen dito en, dito två, dito tre,\\
Så dör du nöjdare.\par
\vspace{10pt}
Du vid din remmare och präss,\\
Rödbrusig och med hatt på sned,\\
Snart skrider fram din likprocess\\
I några svarta led;\\
Och du som pratar där så stort,\\
Med band och stjernor på din rock,\\
Ren snickarn kistan färdig gjort,\\
Och hyflar på des lock.\\
Tycker du...\par
\vspace{10pt}
Men du som med en trumpen min,\\
Bland riglar, galler, järn och lås,\\
Dig hvilar på ditt penningskrin,\\
Innom din stängda bås;\\
Och du som svartsjuk slår i kras\\
Buteljer, speglar och pocal;\\
Bjud nu god natt, drick ut dit glas,\\
Och helsa din rival;\\
Tycker du...\par
\vspace{10pt}
Och du som under titlars klang\\
Din tiggarstaf förgylt hvart år,\\
Som knappast har, med all din rang,\\
En skilling til din bår;\\
Och du som ilsken, feg och lat,\\
Fördömmer vaggan som dig hvälft,\\
Och ändå dagligt är placat\\
Til glasets sista hälft;\\
Tycker du...\\
	\\		  
Du som vid Martis fältbasun\\
I blodig skjorta sträckt ditt steg;\\
Och du som tumlar i paulun,\\
I Chloris armar feg;\\
Och du som med din gyldne bok\\
Vid templets genljud reser dig,\\
Som rister hufvud lärd och klok,\\
Och för mot afgrund krig;\\
Tycker du...\par
\vspace{10pt}
Men du som med en ärlig min\\
Plär dina vänner häda jämt,\\
Och dem förtalar vid dit vin,\\
Och det liksom på skämt;\\
Och du som ej försvarar dem,\\
Fastän ur deras flaskor du,\\
Du väl kan slicka dina fem,\\
Hvad svarar du väl nu?\\
Tycker du...\\  
\\
Men du som til din återfärd,\\
Ifrån det du til bordet gick,\\
Ej klingat för din raska värd,\\
Fastän han ropar: Drick!\\
Drif sådan gäst från mat och vin,\\
Kör honom med sitt anhang ut,\\
Och sen med en ovänlig min,\\
Ryck remmarn ur hans trut.\\
Tycker du...\\
	\\				   
Säg är du nöjd? min granne säg,\\
Så prisa värden nu til slut;\\
Om vi ha en och samma väg,\\
Så följoms åt; drick ut.\\
Men först med vinet rödt och hvitt\\
För vår Värdinna bugom oss,\\
Och halkom sen i grafven fritt,\\
Vid aftonstjernans bloss.\\
Tycker du...
