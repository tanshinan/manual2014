\visa{DU GAMLA, DU FRIA}
\vspace{10pt}
Du gamla, Du fria, Du fjällhöga nord\\
Du tysta, Du glädjerika sköna!\\
Jag hälsar Dig, vänaste land uppå jord,\\
\revrpt Din sol, Din himmel, Dina ängder gröna.\rpt\par
\vspace{10pt}
Du tronar på minnen från fornstora dar,\\
då ärat Ditt namn flög över jorden.\\
Jag vet att Du är och Du blir vad du var.\\
\revrpt Ja, jag vill leva jag vill dö i Norden.\rpt\par
\vspace{10pt}
Jag städs vill dig tjäna mitt älskade land,\\
din trohet till döden vill jag svära.\\
Din rätt, skall jag värna, med håg och med hand,\\
\revrpt din fana, högt den bragderika bära.\rpt\par
\vspace{10pt}
Med Gud skall jag kämpa, för hem och för härd,\\
för Sverige, den kära fosterjorden.\\
Jag byter Dig ej, mot allt i en värld\\
\revrpt Nej, jag vill leva jag vill dö i Norden.\rpt\par
\vspace{10pt}
{\footnotesize\textit{Text: Richard Dybeck, 1844\\
Sveriges nationalsång av tradition sedan 1866.\\
De två sista verserna sjungs sällan.}}
