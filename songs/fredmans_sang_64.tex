\fvisa{Fredmans Sång N:\lowercase{o} 64}{Fjäriln vingad syns på Haga}
{\footnotesize\textit{Haga}}\par
\vspace{10pt}
Fjäriln vingad syns på Haga\\
mellan dimmors frost och dun\\
sig sitt gröna skjul tillaga\\
och i blomman sin paulun.\\
Minsta kräk i kärr och syra,\\
nyss av solens värma väckt,\\
till en ny högtidlig yra\\
eldas vid zefirens fläkt.\par
\vspace{10pt}
Haga, i ditt sköte röjes\\
gräsets brodd och gula plan.\\
Stolt i dina rännlar höjes\\
gungande den vita svan.\\
Längst ur skogens glesa kamrar\\
höres täta återskall,\\
än från den graniten hamrar,\\
än från yx i björk och tall.\par
\vspace{10pt}
Se, Brunnsvikens små najader\\
höja sina gyllne horn,\\
och de frusande kaskader\\
sprutas över Solna torn.\\
Under skygd av välvda stammar\\
på den väg, man städad ser,\\
fålen yvs och hjulet dammar,\\
bonden milt åt Haga ler.\par
\vspace{10pt}
Vad gudomlig lust att röna\\
inom en så ljuvlig park,\\
då man, hälsad av sin sköna,\\
ögnas av en mild monark!\\
Varje blick, hans öga skickar,\\
lockar tacksamhetens tår.\\
Rörd och tjust av dessa blickar,\\
själv den trumpne glättig går.\par
\vspace{10pt}
{\footnotesize\textit{Text \& Musik: Carl Michael Bellman}}
