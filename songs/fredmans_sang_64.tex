\fvisa{Fredmans Sång N:o 64}{Fjäriln vingad syns på Haga}
\vspace{10pt}
Fjäriln vingad syns på Haga,\\
Mellan dimmors frost och dun,\\
Sig sitt gröna skjul tillaga,\\
Och i blomman, sin paulun;\\
Minsta kräk i kärr och syra,\\
Nyss af Solens värma väckt,\\
Til en ny högtidlig yra\\
Eldas vid Zephirens flägt.\\
\\

Haga, i ditt sköte röjes\\
Gräsets brodd och gula plan;\\
Stolt i dina ränlar höjes\\
Gungande den hvita Svan;\\
Längst ur skogens glesa kamrar\\
Höras täta återskall,\\
Än från den graniten hamrar,\\
Än från yx i björk och tall.\\
\\

Se Brunsvikens små Najader\\
Höja sina gyldne horn,\\
Och de frusande cascader\\
Sprutas öfver Solna torn;\\
Under skygd af hvälfda stammar,\\
På den väg man städad ser,\\
Fålen yfs och hjulet dammar,\\
Bonden mildt åt Haga ler.\\
\\

Hvad Gudomlig lust at röna\\
Innom en så ljuflig Park,\\
Då man hälsad af sin Sköna,\\
Ögnas af en mild Monark!\\
Hvarje blick hans öga skickar,\\
Lockar tacksamhetens tår;\\
Rörd och tjust af dessa blickar,\\
Sjelf den trumpne glädtig går.\\
