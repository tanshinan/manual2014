\visa{DEN FÖRSTA GÅNG JAG SÅG DIG}
{\footnotesize\textit{Text och melodi: Birger Sjöberg}}\par
\vspace{10pt}
Den första gång jag såg dig, det var en sommardag\\
på förmiddan, då solen lyste klar,\\
och ängens alla blommor av många hundra slag,\\
de stodo bugande i par vid par.\\
Och vinden drog så saktelig, och nere invid stranden,\\
där smög en bölja kärleksfullt till snäckan uti sanden.\\
Den första gång jag såg dig, det var en sommardag,\\
den första gång jag tog dig uti handen.\par
\vspace{10pt}
Den första gång jag såg dig, då glänste sommarskyn,\\
så bländande som svanen i sin skrud.\\
Då kom det ifrån skogen, från skogens gröna bryn\\
liksom ett jubel utav fåglars ljud.\\
Då ljöd en sång från himmelen, så skön som inga flera;\\
det var den lilla lärkan grå, så svår att observera.\\
Den första gång jag såg dig, då glänste sommarskyn\\
så bländande och grann som aldrig mera.\par
\vspace{10pt}
Och därför när jag ser dig, om ock i vinterns dag,\\
då drivan ligger glittrande och kall,\\
nog hör jag sommarns vindar och lärkans friska slag\\
och vågens brus i alla fulla fall.\\
Nog tycker jag ur dunig bädd sig gröna växter draga\\
med blåklint och med klöverblad, som älskande behaga,\\
att sommarsolen skiner på dina anletsdrag,\\
som rodna och som stråla och betaga.
